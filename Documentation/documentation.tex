\documentclass[pdftex,12pt,a4paper]{report}

% standard packages
\usepackage{graphicx,amsmath,url}
\usepackage{color}


% for easier formatting of the integer variable type
\newcommand{\iv}{(\textit{integer})}
% for easier formatting of the real variable type
\newcommand{\rv}{(\textit{real})}
% for easier formatting of the logical variable type
\newcommand{\lv}{(\textit{logical})}
% for easier formatting of the character variable type
\newcommand{\cv}{(\textit{character})}

\begin{document}
\title{Peeling Ballooning in 3D - Documentation}
\date{2014}
\author{Toon Weyens\\ Departamento de F\'isica, Universidad Carlos III de Madrid}


\maketitle

\chapter{Variables}

\section{PB3D.f90}

\begin{itemize}
 \item \texttt{numargs} \iv: number of arguments passed when calling the program
 \item \texttt{iseq} \iv: index running over all the passed arguments
 \item \texttt{command\_arg(10)} \cv: command line arguments
 \item \texttt{prog\_name} \cv: name of program (PB3D), used when displaying info with \texttt{-h}
\end{itemize}

\section{\texttt{driver\_rich.f90}}

\section{plasma\_vars.f90}

\section{num\_vars.f90}

\begin{itemize}
 \item \texttt{dp} \iv: kind of double precision
 \item \texttt{dx} \rv: discretization parameter
 \item \texttt{rel\_prec} \rv: relative precision criterion for convergence
 \item \texttt{style} \iv: determines method used for minimization
  \begin{itemize}
   \item 1 [\texttt{def}]: Euler-Lagrange minimization of energy, finite differences and Richardson's method
  \end{itemize}

\end{itemize}

\end{document}
